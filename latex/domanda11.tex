\section{Circuiti RC: carica e scarica del condensatore.
	Comportamento alla stazionarietà.}
Per \textbf{Processo di carica} del condensatore si intende l'accumulo di carica sulle armature dello stesso quando \`e applicata una differenza di potenziale $\Delta V$.\\
Allo stesso modo si definisce \textbf{Processo di scarica} del condensatore il rilascio delle cariche precedentemente accumulate, facendo generare allo stesso una differenza di potenziale.\\
\textbf{Fase di carica} :\\
Vale la seguente legge:
$$
    Q(t) = Q_{MAX}\left(1 - e^{-\frac{t}{RC}}\right)
$$
$$
    Q(t) = C \Delta V\left(1 - e^{-\frac{t}{RC}}\right)	 
$$
\begin{itemize}
	\item [$t < 0$] { Condensatore scarico, $\Delta V = 0$ }
	\item [$t \rightarrow 0$] { 
		Condensatore in carica, $\Delta V = \varepsilon$ 
		carica il condensatore : \\
	        $$
	           Q(t) = Q_{MAX}\left(1 - e^{-\frac{0}{RC}}\right) = Q\left(1 - 1\right) = 0
	        $$
    }
    \item [$t \rightarrow \infty$] { 
    	Condensatore carico a regime,
    	$\Delta V(t)$ carica il condensatore:
    	$$
    	    Q(t) = Q_{MAX}\left(1 - e^{-\frac{\infty}{RC}}\right) = 
    	    Q_{MAX}\left(1 - e^{-\infty}\right) = 
    	    Q_{MAX}\left(1 - 0\right) = Q_{MAX} = C \varepsilon
    	$$
    }
\end{itemize}
\textbf{ Fase di scarica } :
Vale la seguente legge :
$$
    Q(t) = C \Delta V \left(e^{-\frac{t}{RC}}\right)
$$
\begin{itemize}
	\item [$t < 0$] {
		Il condensatore \`e carico. 
		$\Delta V = \varepsilon$, 
		$Q(t) = Q_{MAX} = C \varepsilon$ 
	}
	\item [$t \rightarrow 0$] {
        Il condensatore inizia a scaricarsi.\\
        $\Delta V = 0$, $Q(t) = Q_{MAX}\left(e^{-\frac{0}{RC}}\right) = 
        Q_{MAX}\left(e^0\right) = C \varepsilon$\\
        Si comporta come un \textbf{generatore di corrente} :
        $$
        \varepsilon_C = \frac{Q_C}{C}
        $$
	}
	\item [$t \rightarrow \infty$] { 
	    Il condensatore \`e scarico.\\
	    $\Delta V = 0$, 
    	$
    	Q(t) = Q_{MAX}\left(e^{-\frac{\infty}{RC}}\right) = 
    	Q_{MAX}\left(e^{-\infty}\right) = Q_{MAX} \cdot 0 = 0
    	$ 
	}
\end{itemize}
$\hfill\square$