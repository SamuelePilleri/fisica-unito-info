\section{La legge di Coulomb ed il principio di sovrapposizione.
	Come esempio si calcoli il campo di un dipolo elettrico sul
	piano mediano del dipolo oppure lungo l'asse del dipolo.}
La \textbf{Legge di Coulomb} dice che:
\begin{equation}
  \vec{F}_{q_1, q_2} = \frac{1}{4 \pi \varepsilon_0} \frac{q_1 q_2}{r_{1 \rightarrow 2}^{2}}\hat{r}
\end{equation}
Dove:
\begin{itemize}
	\item[$ \varepsilon_0$] {Costante dielettrica nel vuoto}
	\item[$ \hat{u} $] {Versore della forza : le cariche di stesso segno si respingono, viceversa si attraggono}
	\item[$ q_1, q_2 $] {Le cariche di riferimento}
	\item[$ \vec{r_{1, 2}} $] {vettore che congiunge i 2 punti dove si trovano le cariche $ q_1 $ e $ q_2 $}
\end{itemize}
Il \textbf{principio di sovrapposizione} dice che, date \emph{n} cariche in prossimita di $ q_j $, allora vale che:
\begin{equation}
    \vec{F}_{q_j} = \sum_{i \neq j}^{n}{\vec{F}_{q_j, q_i}}
\end{equation}
Con:
$$ 0 <= j < n $$
\\
Si ricorda la formula del campo elettrico prodotto  in un punto $\vec{r}$ da una carica $q$ posizionata in un punto $P$ :
\begin{displaymath}
\vec{E}\left( \vec{r} \right) = 
\frac{1}{4 \pi \varepsilon_0} \frac{q}{|\vec{r}|^2} \hat{r} = 
\frac{1}{4 \pi \varepsilon_0} \frac{\vec{F}_{coulomb} (q, q_{test})}{q_{test}} 
\end{displaymath}
Grazie al principio di sovrapposizione prima citato possiamo affermare che, date due cariche $+q$ e $-q$ :
\begin{equation}
\vec{E}\left( \vec{r} \right) = 
\vec{E}_{q_1}\left( \vec{r} \right) + \vec{E}_{q_2}\left( \vec{r} \right)
\end{equation}

\begin{center}
    \textbf{(per il  campo elettrico sull'asse del dipolo vedere domanda 5 Esempio 1)}
\end{center}
