\section{Descrivere le equazioni del moto di una carica elettrica
	in un campo magnetico uniforme.}
Una carica elettrica immersa in un campo magnetico \`e soggetta a una forza (chiamata \textbf{Forza di Lorentz}) cosi definita:
$$
    \vec{F} =  q\vec{v} \times \vec{B}
$$
Dove:
\begin{itemize}
	\item [$q$] { La carica elettrica. }
	\item [$\vec{v}$] { La velocit\'a del moto della carica. }
	\item [$\vec{B}$] { Il campo magnetico in cui la carica \`e immersa. }
\end{itemize}
A seconda dei casi il contributo \`e differente :
\begin{itemize}
	\item [caso $\vec{v} \parallel \vec{B}$] { Il prodotto vettoriale \`e nullo, pertanto la particella non \`e soggetta a forza magnetica. }
	\item [caso $\vec{v} \perp \vec{B}$] { Il prodotto vettoriale \`e massimizzato e di direzione perpendicolare al piano contenente $\vec{v}$ e $\vec{B}$. Per il verso si usa la regola della mano destra.
		Nel caso $B$ sia costante si ha un moto su traiettoria circolare. }
	\item [caso : moto a elica] { 
		Si scompone il vettore $\vec{v}$ in 2 componenti perpendicolari fra loro : \\
		$\vec{v} = \vec{v}_{\parallel} + \vec{v}_{\perp}$ dove:
		\begin{itemize}
			\item [$\bullet$] { 
				$\vec{v}_{\parallel}$ non da alcun contributo alla forza.
			}
			\item [$\bullet$] { $\vec{v}_{\perp}$ produce una forza ortogonale al piano contenente $\vec{v}$ e $\vec{B}$. }
		\end{itemize}
		La combinazione dei due moti genera un moto elicoidale avente asse coincidente con $\hat{u}_B$ 
	}
\end{itemize}
$\hfill\square$